\documentclass{beamer}	% Mudança na classe documento

\usepackage[brazilian]{babel}
\usepackage[utf8]{inputenc}

% Modo de apresentação
\mode<presentation>
{
  \usetheme{default}
  \usecolortheme{default}
  \usefonttheme{default}
  \setbeamertemplate{navigation symbols}{}
  \setbeamertemplate{caption}[numbered]
}

% Título, autor, instituto, data ... bem parecido com a classe de documento 
\title[Your Short Title]{Your Presentation}
\author{You}
\institute{Where You're From}
\date{Date of Presentation}

\begin{document}

% Utilizar begin frame e end frame para definir um slide
\begin{frame}
  \titlepage
\end{frame}

\section{Introduction}

\begin{frame}{Introduction}

\begin{itemize}
  \item Olá eu sou um item
\end{itemize}

% Utilizar begin block e end block para definir um bloco no slide
\begin{block}{Exemplo}
Aqui tem a explicação de "Exemplo"
\end{block}

\end{frame}

\end{document}