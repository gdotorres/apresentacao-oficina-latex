\documentclass[12pt,preview]{standalone}

\usepackage[english]{babel}
\usepackage[utf8]{inputenc} % acentuacao
\usepackage[T1]{fontenc} % para permitir copiar corretamente de PDFs para textos

\usepackage{hyperref}

\input{../preamble/fp.tex}
\input{../preamble/common.tex} 
\input{../preamble/color.tex}
\input{../preamble/todo.tex}
\input{../preamble/charts.tex}

\begin{document}

% Pre-plot stuff
% We will create an inline table called \datatable from memReqs.csv. We'll only read rows which have readReqs > 50k
\pgfplotstablefilterread[col sep=comma]{../csvs/memReqs.csv}{benchmark}{readReqs}{< 50000}{\datatable}

% Plot stuff
\begin{tikzpicture}

\begin{axis}[
% Plot style
ybar,
% Size options
width=\textwidth,height=7.5cm,
% Limits of plot
ymin=0,ymax=40000,
enlarge x limits=0.025,
% Bar options
bar width=6pt,
% Grid options
ymajorgrids=true,yminorgrids=true,
yminorticks=true,minor y tick num=1,
% X ticks
flexible xticklabels={\datatable}{benchmark},
x tick label style={
	rotate=45, 
	font=\ttfamily\scriptsize,
	anchor=north east, 
	inner sep=0mm,
},
xtick=data,
% Y ticks
yticklabel={\pgfmathparse{\tick}\pgfmathprintnumber{\pgfmathresult}},
% Axis labels
ylabel=Main memory generated read requests,
]
\addplot+[black,fill=white] table[x expr=\coordindex,y=readReqs]{\datatable};
\end{axis}

\end{tikzpicture}

\end{document}