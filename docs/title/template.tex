% Portuguese language
%\documentclass[report,noindex,lean]{enacom}
%\documentclass[notes,noindex]{enacom}
%\documentclass[thesis,noindex]{enacom}
%\documentclass[book,noindex]{enacom}
%\documentclass[report,portuguese,noindex,lean]{enacom}

% English language
\documentclass[report,english]{enacom} 
%\documentclass[notes,english]{enacom}
%\documentclass[thesis,english]{enacom}
%\documentclass[book,english]{enacom}
%\documentclass[report,english,noindex,lean]{enacom}

\type{Document Type}

\title{Title}
\subtitle{Subtitle}

\allowdisplaybreaks

%\partner{UFMG}{ufmg}

\author[~][degree]{Name}{Surname}
\author{Name}{Surname}
\author{Name}{Surname}
\author{Name}{Surname}

\support{UFMG}{ufmg}

\local{Belo Horizonte}

\date{\today}

\bib{template}


\copyrights{The contents of this document is copyrighted by ENACOM. No portion of the content may be directly or indirectly copied, published, reproduced, modified, performed, displayed, sold, transmitted, published, broadcast, rewritten for broadcast or publication or redistributed in any medium.}

\update{DD/MM/YYYY}{Author Name}{\item initial version}
\update{DD/MM/YYYY}{Author Name}{\item update}
\update{DD/MM/YYYY}{Author Name}{\item update 1\item update 2 \item update 3}

\symbols{
	$\class{C}$ & a class\\
	$\set{R}$ & a set\\
	$M$ & a matrix\\
	$v$ & a vector\\
}

\abbreviations{
	ANSI & American National Standards Institutean abbreviation\\
}

\begin{document}

\chapter{This is a chapter}


\section{Template overview}


\subsection{Document options}
	\begin{itemize}
		\item document type:
			\begin{itemize}
				\item \code{accept}
				\item \code{article}
				\item \code{book}
				\item \code{handout}
				\item \code{hh}
				\item \code{letter}
				\item \code{notes}
				\item \code{poster}
				\item \code{record}
				\item \code{report} (default)
				\item \code{slides}
				\item \code{thesis}
			\end{itemize}
	\end{itemize}
	\begin{itemize}
		\item language:
			\begin{itemize}
				\item \code{english}
				\item \code{portuguese} (default)
			\end{itemize}
		\item font size:
			\begin{itemize}
				\item \code{10pt}
				\item \code{11pt} (default)
				\item \code{12pt}
			\end{itemize}
		\item color:
			\begin{itemize}
				\item \code{colorful} (default)
				\item \code{grayscale}
			\end{itemize}
	\end{itemize}


\subsection{Title items}
	\begin{itemize}
		\item \begin{lcode} 
        \type{<the document type name>} 
        \end{lcode} 
        
        \item \begin{lcode}
      	\title{<the title>}\end{lcode}
        
        \item \begin{lcode}
      	\subtitle{<the subtitle>}\end{lcode}
        
        \item \begin{lcode}
        \author[<position>]{<name>}{<surname>} \end{lcode}
        
        \item \begin{lcode}
        \advisor[<position>]{<name>}{<surname>} \end{lcode}
        
        \item \begin{lcode}
        \partner{<name>}{<figure>} \end{lcode}
        
        \item \begin{lcode} 
        \support{<name>}{<figure>} \end{lcode}
        
        \item \begin{lcode}
        \date{<the date>} \end{lcode}
        
        \item \begin{lcode}
        \local{<the local>} \end{lcode}
	\end{itemize}


\subsection{Document items}
	\begin{itemize}
    	\item \begin{lcode} 
        \copyrights{<the copyrights text>} \end{lcode}
        
        \item \begin{lcode} 
        \abstract{<the abstract text>} \end{lcode}
        
        \item \begin{lcode} 
        \ack{<the extra acknowledgement text>} \end{lcode}
        
        \item \begin{lcode}
        \bib{<the bib-file name>} \end{lcode}
    
        \item \begin{lcode}
        \update{<the date>}{<the author name>}{
        \item <the update 1 brief description>
        \item <the update 2 brief description>
        ...} \end{lcode}
    
        \item \begin{lcode}
        \symbols{
        $<symbol>$ & <meaning>\\
        $<symbol>$ & <meaning>\\
        ...} \end{lcode}

        \item \begin{lcode} 
        \abbreviations{
        <abbreviation> & <meaning>\\
        <abbreviation> & <meaning>\\
        ...} \end{lcode}
    \end{itemize}
    

\subsection{Options for cleaning up document}
	\begin{itemize}
		\item \code{nobackpage} for back page removal;
		\item \code{nosummary} for summary page removal;
		\item \code{nocopyright} for copyright text removal;
		\item \code{noupdate} for update history section removal;
		\item \code{noindex} for remissive index section removal;
		\item \code{lean} for blank pages removal;
		\item \code{nofiglist} for list of figures removal;
		\item \code{notablist} for list of tables removal.
	\end{itemize}


\section{LaTeX  elements}


\subsection{Sectioning}
	\begin{itemize}
		\item \begin{lcode}
        \chapter{<chapter name>} \end{lcode}
        
        \item \begin{lcode}
        \section{<section name>} \end{lcode}
        
        \item \begin{lcode}
        \subsection{<subsection name>} \end{lcode}
	
    	\item \begin{lcode}
        \subsubsection{<subsubsection name>}
        \end{lcode}
	
    	\item \begin{lcode} 
        \paragraph{<paragraph name>} \end{lcode}
	\end{itemize}
    

\subsection{List Structures}

\subsubsection{Enumeration}
	The \LaTeX code
    \begin{lcode}
	\begin{enumerate}
    	\item first
    	\begin{enumerate}
    		\item first first
			\begin{enumerate}
				\item first first first
			\end{enumerate}
		\end{enumerate}
		\item second
	\end{enumerate} 
    \end{lcode}
	%
	results in
	%
	\begin{enumerate}
    \item first
    \begin{enumerate}
    \item first first
    \begin{enumerate}
				\item first first first
			\end{enumerate}
		\end{enumerate}
		\item second
	\end{enumerate}

\subsubsection{Description}
    The \LaTeX code
        \begin{lcode}
        \begin{description}
            \item [item] description
            \item [item] description
        \end{description}
        \end{lcode}
        %
        results in
        %
    \begin{description}
        \item[item] description
        \item[item] description
    \end{description}

\subsubsection{Itemization}
	The \LaTeX code
	\begin{lcode}
	\begin{itemize}
		\item item
		\begin{itemize}
			\item subitem
			\begin{itemize}
				\item subsubitem
			\end{itemize}
		\end{itemize}
		\item item
	\end{itemize}
	\end{lcode}
	%
	results in
	%
	\begin{itemize}
		\item item
		\begin{itemize}
			\item subitem
			\begin{itemize}
				\item subsubitem
			\end{itemize}
		\end{itemize}
		\item item
	\end{itemize}


\subsection{Theorems and proofs}

\subsubsection{Definition}
	The \LaTeX code
	\begin{lcode}
	\begin{definition}[something]
    	This is the definition of something.
	\end{definition}
	\end{lcode}
    %
    results  in		
    %
    \begin{definition}[something]
    	This is the definition of something.
	\end{definition}

\subsubsection{Theorem}
	The \LaTeX code
    \begin{lcode}
    \begin{theorem}[someone]
    	This is the statement of someone's theorem.
    \end{theorem}
    \begin{proof}
    	This is the proof of someone's theorem.
    \end{proof}
    \end{lcode}
    %
    results in
    %
    \begin{theorem}[someone]
    	This is the statement of someone's theorem.
    \end{theorem}
    \begin{proof}
    	This is the proof of someone's theorem.
    \end{proof}

\subsubsection{Lemma}
	The \LaTeX code
    \begin{lcode}
    \begin{lemma}[someone]
    	This is the statement of someone's lemma.
    \end{lemma}
    \begin{proof}
    	This is the proof of someone's lemma.
    \end{proof}
    \end{lcode}
    %
    results in
    %
    \begin{lemma}[someone]
    	This is the statement of someone's lemma.
    \end{lemma}
    \begin{proof}
    	This is the proof of someone's lemma.
    \end{proof}

\subsubsection{Corollary}
	The \LaTeX code
	\begin{lcode}
	\begin{corollary}[someone]
		This is the statement of someone's corollary.
	\end{corollary}
	\end{lcode}
	%
	results in
    %
	\begin{corollary}[someone]
	This is the statement of someone's corollary.
	\end{corollary}


\subsection{Footnote}
	Foot notes are created with command \code{footnote} and they are reference by a superscripted number\footnote{This is a foot note. It is always positioned on the bottom of the column and page where its reference occurs. Long foot notes may have more than one text line.}.


\subsection{Equations}
    \begin{itemize}
        \item use \code{equation} or \code{align} to place a numbered equation;
        \begin{equation}\label{eq.Series}
            f(x) = x_1 + \frac{x_3^3}{3} + \frac{x_5^5}{5};
        \end{equation}		
        \item use command \code{nonumber} to unnumber equations;
        \item use command \code{label} to assign a label to an equation;
        \begin{align}
            \mi & f(x)\label{eq.eq1}\\
        \sto & g(x) \leq 0\label{eq.eq2}\\
            & h(x) = 0\label{eq.eq3}\\
            & x \in \set{R}^n\label{eq.eq4};
        \end{align}
        \item use command \code{eqref} or \code{autoref} to refer to a numbered equation through its label:\\
        Example \code{eqref}: \eqref{eq.eq1}. \\
        Example \code{autoref}: \autoref{eq.eq1}.  
    \end{itemize}


\subsection{Table}
	\begin{itemize}
		\item use command \code{tabular} to insert a table;
		\item use environment \code{table} to support caption and references;
		\begin{itemize}
			\item use command \code{caption} to write a table caption;
			\item use command \code{label} to assign a label to a table;
		\end{itemize}
	\begin{table}[H]				
    	\rowbicolor{3} % argument: starting row
		\begin{tabular}{m{4cm}cc}
			\multirow{2}{*}{activity} & 
			\multicolumn{2}{c}{month}\\ & 1 & 2 \\ \hline
			first \linebreak
			after line break & x &   \\
			second: two lines due to width
			&   & x \\
		    third  & x &   \\
			fourth &   & x \\
		\end{tabular}
		\caption{A table.}
		\label{tab.Table}
	\end{table}
    \item use command \code{autoref} to refer to a table through its label:\\
    Example \code{autoref}: \autoref{tab.Table}.
\end{itemize}


\subsection{Text styles}
    Special text mode set commands, 
    (\autoref{tab.MathTextCommands}).
        \begin{table}[H]
        \begin{tabular}{rr}
            code & result \\
        \hline
             \begin{lcode}
             \qm{a quoted tex}\end{lcode} 
             & \qm{a quoted tex}\\
             \begin{lcode}
             \code{a code text}\end{lcode} 
             & \code{a code text}\\
             \begin{lcode}
             \textit{an italic text}\end{lcode}
             & \textit{an italic text}\\
             \begin{lcode}
             \textbf{a bold face text}\end{lcode} 
             & \textbf{a bold face text}\\
             \begin{lcode}
             \textbackslash\end{lcode} &\textbackslash\\
             \begin{lcode}
             \%\end{lcode} &\%\\
             \begin{lcode}
             \$\end{lcode} &\$\\
             \begin{lcode}
             \&\end{lcode} &\&\\
        \end{tabular}
            \caption{Special text set commands.}
            \label{tab.MathTextCommands}
        \end{table}


\newpage
\subsection{Math commands}

\subsubsection{Special math set commands}
    \begin{table}[H]
		\begin{tabular}{rr}
			code & result \\
			\hline
		 	\begin{lcode}
       		\set{R}\end{lcode} & $\set{R}$\\
		 	\begin{lcode}
       		\class{G}\end{lcode} & $\class{G}$\\
		 	\begin{lcode}
       		\nin\end{lcode} & $\nin$\\
		 	\begin{lcode}
       		\card{\set{S}}\end{lcode} &
            $\card{\set{S}}$\\
            \begin{lcode}
       		\floor{n}\end{lcode} & 
       		$\floor{n}$\\
		 	\begin{lcode}
       		\ceil{n}\end{lcode} 
            & $\ceil{n}$\\
       \end{tabular} 
	\caption{Special math set commands.}
	\end{table}
    
\subsubsection{Special math matrix commands}
    \begin{table}[H]
		\begin{tabular}{rr}
    		code & result \\ \hline
       		\begin{lcode}
       		\T{M}\end{lcode} & $\T{M}$\\
		 	\begin{lcode}
       		\inv{M}\end{lcode} 
            & $\inv{M}$\\
		 	\begin{lcode}
       		\invT{M}\end{lcode} 
            & $\invT{M}$\\
		 	\begin{lcode}
       		\diag(M)\end{lcode} 
            & $\diag(M)$\\
       \end{tabular} 
	\caption{Special math matrix commands.}
	\end{table}
    
\subsubsection{Special math function commands} 
    \begin{table}[H]
		\begin{tabular}{rr}
            code & result \\ \hline
            \begin{lcode}
            \e^\pi\end{lcode} 
            & $\e^\pi$\\
            \begin{lcode}
            \gradient f\end{lcode} 
            & $\gradient f$\\
            \begin{lcode}
            \hessian f\end{lcode} 
            & $\hessian f$\\
             \begin{lcode}
            \mi f(x)\end{lcode} 
            & $\mi f(x)$\\
            \begin{lcode}
            \ma f(x)\end{lcode} 
            & $\ma f(x)$\\
            \begin{lcode}
            \sto g(x) \leq 0\end{lcode} 
            & $\sto g(x) \leq 0$\\
		\end{tabular} 
		\caption{Special math function commands.}
	\end{table}
    
\subsubsection{Special math vector or complex commands}
	\begin{table}[H]
		\begin{tabular}{rr}
        	code & result \\ \hline
            \begin{lcode}
            \opt{x}\end{lcode} & $\opt{x}$\\
            \begin{lcode}
            \conj{z}\end{lcode} 
            & $\conj{z}$\\
            \begin{lcode}
            \real(z)\end{lcode} 
            & $\real(z)$\\
            \begin{lcode}
            \imag(z)\end{lcode} & $\imag(z)$\\
            \begin{lcode}
            \abs{z}\end{lcode} & $\abs{z}$\\
            \begin{lcode}
            \norm{v}\end{lcode} & $\norm{v}$\\
            \begin{lcode}
            \mean_i v_i\end{lcode} 
            & $\mean_i v_i$\\
            \begin{lcode}
            \dsum_{i=1}^n v_i\end{lcode}
            & $\dsum_{i=1}^n v_i$\\
            \begin{lcode}
            \dprod_{i=1}^n v_i\end{lcode} 
            & $\dprod_{i=1}^n v_i$\\
            \end{tabular} 
	\caption{Special math vector or complex commands.}
	\end{table}
    

\subsection{Figure}
    \begin{itemize}
        \item use command \code{includegraphics} to insert a figure;
        \begin{itemize}
            \item no need to use file extensions;
            \item supported files: PDF, EPS, PNG and JPG (search in this order);
        \end{itemize}
        \item use environment \code{figure} to support caption and references;
        \begin{itemize}
            \item use command \code{caption} to write a figure caption;
            \item use command \code{label} to assign a label to a figure;
        \end{itemize}
    \end{itemize}
    %
    \begin{figure}[h!]
        \includegraphics[scale=0.5]{enacomfig.pdf}
        \caption{Figure 1}
        \label{fig.figure1}
    \end{figure}
    %
    \begin{figure}[h!]
    	\subfloat[figure 1]
    	{\label{fig.vector}\includegraphics[width=5cm]{enacomfig.pdf}}
    	\subfloat[Figure 2]
    	{\label{fig.raster}\includegraphics[width=5cm]{enacomfig.pdf}}
    	\caption{Figure 2}
    \end{figure}
    %
    \begin{itemize}
        \item use command \code{autoref} to refer to a figure through its label: \\
    	Example: \autoref{fig.figure1}.
    \end{itemize}


\subsection{Codes}

\subsubsection{Python code}
    \begin{itemize}
        \item use \code{lstlisting} for Python code
    \end{itemize}
    Writing code in \LaTeX~ document
    \begin{lcode}
    \begin{lstlisting}[language=python]
        count = 0
        while count < 5:
        print(count)
        count += 1  # comment
    \end{lstlisting}
    \end{lcode}
    %
    results in
    %
    \begin{lstlisting}[language=python]
        count = 0
        while count < 5:
            print(count)
            count += 1  # comment
    \end{lstlisting}

\subsubsection{Matlab Code}
    \begin{itemize}
        \item use \code{mcode} for MATLAB code listings
    \end{itemize}
    Writing code in \LaTeX ~ document
    \begin{lcode}
    \begin{mcode}
        function y = average(x)
            if ~isvector(x)
                error('Input must be a vector')
            end
            y = sum(x)/length(x); 
        end
    \end{mcode}
    \end{lcode}
    %
    results in
    %
    \begin{mcode}
	    function y = average(x)
    	if ~isvector(x)
        	error('Input must be a vector')
    	end
    	y = sum(x)/length(x); 
    	end
    \end{mcode}


\subsection{Algorithm}
	\begin{itemize}
		\item environments:
		\begin{itemize}
			\item use \code{algorithm} to encapsulate input, output and code;
			\item use \code{algorithmic} to encapsulate code.
		\end{itemize}
		\item commands:
		\begin{itemize}
			\item use \code{State} to start a new algorithm line;
			\item use \code{Comment} to place a line comment;
			\item use \code{gets} for attributions.
		\end{itemize}
		\item keywords:
		\begin{itemize}
			\item \code{For}, \code{EndFor};
			\item \code{If}, \code{Else}, \code{EndIf};
			\item \code{Return}, \code{Break};
            \code{Continue}.
		\end{itemize}
	\end{itemize}
    %
    The \LaTeX code
    %
	\begin{lcode}
	\begin{algorithm}
		\caption{Evaluation of sinus of a sum.}
		\label{alg.Sinus}
		\algorithminput{$a$ & first part \\ $b$ & second part\\}
		\algorithmoutput{$s$ & sum of the two parts \\ $t$ & sinus of the sum\\}
        \begin{algorithmic}[1]
        \State $s \gets a + b$ \Comment{sum of input arguments}
        \State $t \gets 0$
        \For{$i = 1, 2...$}
    	    \State $t \gets t + (-1)^{i+1}\frac{s^{2i-1}}{(2i-1)!}$ \Comment{Taylor series for sinus}
        \EndFor
        \State \Return $s$ and $t$
        \end{algorithmic}
	\end{algorithm}
	\end{lcode}
	%
    results in
    %
	\begin{algorithm}
		\caption{Evaluation of sinus of a sum.}
		\label{alg.Sinus}
		\algorithminput{$a$ & first part \\ $b$ & second part\\}
		\algorithmoutput{$s$ & sum of the two parts \\ $t$ & sinus of the sum\\}
			\begin{algorithmic}[1]
			\State $s \gets a + b$ \Comment{sum of input arguments}
			\State $t \gets 0$
			\For{$i = 1, 2...$}
				\State $t \gets t + (-1)^{i+1}\frac{s^{2i-1}}{(2i-1)!}$ \Comment{Taylor series for sinus}
			\EndFor
			\State \Return $s$ and $t$
		\end{algorithmic}
	\end{algorithm}


\begin{landscape}
    \subsection{Page orientation}
        This is a page in landscape. The code for this is:
        \begin{lcode}
        \begin{landscape}
            \subsubsection{Page orientation}
            This is a page in landscape. The code for this is:
        \end{landscape}
        \end{lcode}
\end{landscape}


\subsection{Bibliography}
	\begin{itemize}
		\item use command \code{bib} in preamble to specify bib-file;	
        \item use command \code{cite} to cite a reference as their authors;\\	
        \cite{Article}.\\
        \cite{Book}.
        \item use command \code{citep} to cite a reference as a bracket;\\
        \citep{Article}.\\
        \citep{Book}.
        \item separate adjacent citations by commas;\\
        \citep{Article,Book}.
	\end{itemize}

\end{document}