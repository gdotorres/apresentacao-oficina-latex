\section{Formatação básica}

\begin{frame}[fragile] \frametitle{Formatação básica}
\begin{itemize}
	\item Somente texto cru não faz muita coisa
	\item Vários elementos de formatação também se fazem presentes no Latex
\end{itemize}
\end{frame}

\begin{frame}[fragile] \frametitle{Negrito, itálico e sublinhado}
\begin{figure}[!t]
\caption{Exemplo de texto negrito}
\begin{lstlisting}
Eis que você está digitando um texto e deseja fazer um destaque. @*\sel{$\backslash$textbf\{Bom, aqui estão algumas palavras em negrito\}}*@.
\end{lstlisting}
\ownsrc
\end{figure}

\begin{itemize}
	\item \sel{$\backslash$textbf\{texto\}}
	\begin{itemize}
		\item Formata texto em negrito
	\end{itemize}
\end{itemize}
\end{frame}

\begin{frame}[fragile] \frametitle{Negrito, itálico e sublinhado}
\begin{figure}[!t]
\caption{Exemplo de texto itálico}
\begin{lstlisting}
Usualmente, utiliza-se texto em itálico para destacar palavras estrangeiras, como @*\sel{$\backslash$textit\{smartphone\}}*@.
\end{lstlisting}
\ownsrc
\end{figure}

\begin{itemize}
	\item \sel{$\backslash$textit\{texto\}}
	\begin{itemize}
		\item Formata texto em itálico
	\end{itemize}
\end{itemize}
\end{frame}

\begin{frame}[fragile] \frametitle{Negrito, itálico e sublinhado}
\begin{figure}[!t]
\caption{Exemplo de texto sublinhado}
\begin{lstlisting}
E dá pra @*\sel{$\backslash$underline\{deixar elementos sublinhados\}}*@.
\end{lstlisting}
\ownsrc
\end{figure}

\begin{itemize}
	\item \sel{$\backslash$underline\{texto\}}
	\begin{itemize}
		\item Formata texto em sublinhado
	\end{itemize}
\end{itemize}
\end{frame}

\begin{frame}[fragile] \frametitle{Negrito, itálico e sublinhado}
\begin{figure}[!t]
\caption{Exemplo de comandos de formatação aninhados}
\begin{lstlisting}
E o que acontece se fizermos uma \textbf{\textit{salada}} \textit{\underline{mista}}?
\end{lstlisting}
\ownsrc
\end{figure}

\begin{itemize}
	\item Quando fazemos $\backslash$\texttt{comando}, estamos invocando um comando.
	\item Podemos invocar comandos aninhados
	\begin{itemize}
		\item \texttt{$\backslash$comando1\{$\backslash$comando2\{texto\}\}}
	\end{itemize}
	\item Primeiro destaque: Negrito e itálico
	\item Segundo destaque: Itálico e sublinhado
\end{itemize}
\end{frame}