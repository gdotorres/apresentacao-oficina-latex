\section{Classes}

\begin{frame} \frametitle{Classes}
\begin{itemize}
	\item Para um arquivo ser formatado corretamente, faz-se uso de classes
	\item Algumas classes pré-definidas
	\begin{itemize}
		\item \texttt{article, book}, \ldots
	\end{itemize}
	\item Classes construídas por usuários
	\begin{itemize}
		\item Arquivo .cls
	\end{itemize}
\end{itemize}
\end{frame}

\begin{frame} \frametitle{Arquivo .cls}
\begin{itemize}
	\item Define uma classe de arquivo a qual representa uma formatação definida
	\item Utiliza a linguagem TeX para desenvolver
	\begin{itemize}
		\item Linguagem complexa e de difícil entendimento
	\end{itemize}
\end{itemize}
\end{frame}

\begin{frame} \frametitle{Classe do TCC do TSI-IFSul}
\begin{itemize}
	\item \textbf{Caso de exemplo:} Classe do documento de TCC do TSI-IFSul Campus Pelotas
	\begin{itemize}
		\item Segue a mesma formatação do modelo de documento fornecido em .doc
	\end{itemize}
	\item Disponível em um repositório do GitHub:
	\begin{itemize}
		\item \url{https://github.com/gdotorres/tsi-ifsul-tcc}
	\end{itemize}
\end{itemize}
\end{frame}

\begin{frame} \frametitle{Classe do TCC do TSI-IFSul}
\begin{itemize}
	\item Arquivo \textbf{textsi.cls}: Define a formatação geral do documento
	\begin{itemize}
		\item Margens, fonte, espaçamento, \ldots
		\item Define usos de alguns pacotes
		\begin{itemize}
			\item Requeridos para o desenvolvimento desta classe
		\end{itemize}
		\item Desenvolvido na linguagem TeX
	\end{itemize}
	\item Arquivo \textbf{exemplo-tcc.tex}: Pacotes adicionais e corpo do documento
	\begin{itemize}
		\item Focaremos principalmente neste arquivo
	\end{itemize}
	\item Arquivo \textbf{exemplo-tcc.bib}: Referências bibliográficas
	\begin{itemize}
		\item Será visto com mais detalhes adiante
	\end{itemize}
\end{itemize}
\end{frame}

\begin{frame} \frametitle{Classe do TCC do TSI-IFSul -- Entendimento}
\begin{itemize}
	\item \textbf{exemplo-tcc.tex} é composto de conteúdo com comentários indicando para que serve cada linha ou conjunto de linhas de comando
\end{itemize}
\end{frame}

\begin{frame}[fragile] \frametitle{Comentários?}

\begin{lstlisting}
@*\sel{\% Começo de um documento .tex}*@
\documentclass{article} 
(...)
\end{lstlisting}

\begin{itemize}
	\item Estamos utilizando uma linguagem de programação para descrever nosso documento
	\item Como linguagem de programação, ela fornece meio para comentários
	\begin{itemize}
		\item Utilizamos o caractere \%
		\item Somente uma nova linha no código para desfazer o comentário
		\item \textbf{Não} existe comentário em bloco
	\end{itemize}
\end{itemize}
\end{frame}