\section{Ambiente}

\begin{frame} \frametitle{Iniciando um documento no formato .tex}
\begin{itemize}
	\item Criar um arquivo .tex
	\item Deve ser necessário alguns comandos
\end{itemize}
\end{frame}

\begin{frame}[fragile] \frametitle{Iniciando um documento no formato .tex}
\begin{figure}[!t]
\caption{Exemplo de "Olá mundo"~em Latex}
\begin{lstlisting}
\documentclass{article}

\begin{document}
	Ola mundo
\end{document}
\end{lstlisting}
\ownsrc
\end{figure}
\end{frame}

\begin{frame}[fragile] \frametitle{Iniciando um documento no formato .tex}
\begin{figure}[!t]
\caption{Exemplo de "Olá mundo"~em Latex}
\begin{lstlisting}
@*\sel{$\backslash$documentclass\{article\}}*@

\begin{document}
	Ola mundo
\end{document}
\end{lstlisting}
\ownsrc
\end{figure}

\begin{itemize}
	\item \sel{$\backslash$documentclass\{article\}}
	\begin{itemize}
		\item Define a classe de documento que será utilizada
		\item \texttt{article}: Artigo
		\item Existem diversas opções que permite diferentes formatos de documentos
	\end{itemize}
\end{itemize}

\end{frame}

% Exemplos de documentos...

\begin{frame}[fragile] \frametitle{Iniciando um documento no formato .tex}
\begin{figure}[!t]
\caption{Exemplo de "Olá mundo"~em Latex}
\begin{lstlisting}
\documentclass{article}

@*\sel{$\backslash$begin\{document\}}*@
	Ola mundo
@*\sel{$\backslash$end\{document\}}*@
\end{lstlisting}
\ownsrc
\end{figure}

\begin{itemize}
	\item \sel{$\backslash$begin\{document\}} e \sel{$\backslash$end\{document\}}
	\begin{itemize}
		\item Delimita o escopo de conteúdo do documento
		\begin{itemize}
			\item Também chamado de corpo do documento
		\end{itemize}
		\item O que vem antes de \texttt{$\backslash$begin\{document\}} é chamado de \textbf{preâmbulo} do documento
	\end{itemize}
\end{itemize}
\end{frame}

\begin{frame} \frametitle{Preâmbulo do documento}
\begin{itemize}
	\item Primeira metade de um arquivo .tex
	\begin{itemize}
		\item \textbf{Sempre} antes do texto que irá compor o documento
	\end{itemize}
	\item Múltiplas funcionalidades
	\begin{itemize}
		\item Tipo de documento
		\item Informações para formatar o documento corretamente
		\item Carregamento de pacotes que auxiliam para algumas especificidades
	\end{itemize}
\end{itemize}
\end{frame}

\begin{frame} \frametitle{Corpo do documento}
\begin{itemize}
	\item Segunda metade do arquivo .tex
	\item Contém todas as informações referentes a conteúdo do documento
	\begin{itemize}
		\item Texto bruto
		\item Comandos para formatação de texto
		\item Inserção de elementos adicionais
		\begin{itemize}
			\item Figuras, tabelas, fórmulas matemáticas, \ldots
		\end{itemize}
	\end{itemize}
\end{itemize}
\end{frame}

\begin{frame} \frametitle{Localidade}
\begin{itemize}
	\item Por padrão, o~Latex~por si só não é suficiente
	\item Precisamos estabelecer a localidade para a escrita correta de textos
	\item \textbf{Caso de exemplo:} Documento simples com a visualização do texto 
	\begin{itemize}
		\item Pasta \textbf{examples}: Pasta 01\_localizacao\_incorreta
	\end{itemize}
\end{itemize}
\end{frame}

\begin{frame} \frametitle{Localidade}
\begin{itemize}
	\item Adaptar o Latex~para a localidade que desejamos
	\begin{itemize}
		\item Suporte a diversas linguagens espalhadas pelo globo
	\end{itemize}
	\item Modificar para o Latex conseguir interpretar os acentos corretamente -- além de alterar a localidade para português brasileiro.
	\begin{itemize}
		\item \textbf{Preâmbulo} do documento
	\end{itemize}
\end{itemize}
\end{frame}

\begin{frame}[fragile] \frametitle{Localidade}
\begin{figure}[!t]
\caption{Inserindo localidade em Latex}
\begin{lstlisting}
\documentclass{article}

@*\sel{$\backslash$usepackage[brazil]\{babel\}}*@		% Comando para pôr a localidade português brasileiro 
\usepackage[utf8]{inputenc} 	% Comando para reconhecer entradas utilizando a codificação UTF-8

\begin{document}
(@*\ldots*@)
\end{lstlisting}
\ownsrc
\end{figure}

\begin{itemize}
	\item \sel{$\backslash$usepackage[brazil]\{babel\}}
	\begin{itemize}
		\item Datas e palavras fornecidas pela formatação do Latex~são traduzidas para português brasileiro
		\item Se desejável, pode-se trocar a opção entre colchetes para a linguagem que quiser
		\begin{itemize}
			\item \texttt{english}, \texttt{spanish}, \ldots
		\end{itemize}
	\end{itemize}
\end{itemize}
\end{frame}

\begin{frame}[fragile] \frametitle{Localidade}
\begin{figure}[!t]
\caption{Inserindo localidade em Latex}
\begin{lstlisting}
\documentclass{article}

\usepackage[brazil]{babel}		% Comando para pôr a localidade português brasileiro
@*\sel{$\backslash$usepackage[utf8]\{inputenc\}}*@ 	% Comando para reconhecer entradas utilizando a codificação UTF-8

\begin{document}
(@*\ldots*@)
\end{lstlisting}
\ownsrc
\end{figure}

\begin{itemize}
	\item \sel{$\backslash$usepackage[utf8]\{inputenc\}}
	\begin{itemize}
		\item Mostra corretamente caracteres com acento, cedilha, dentre outros
		\item Tecnicamente, faz com que a codificação lida como entrada pelo Latex seja UTF-8, a qual tem suporte a vários caracteres (inclusos os utilizados no português)
	\end{itemize}
\end{itemize}
\end{frame}

\begin{frame}[fragile] \frametitle{Localidade}
\begin{figure}[!t]
\caption{Inserindo localidade em Latex}
\begin{lstlisting}
\documentclass{article}

\usepackage[brazil]{babel}		% Comando para pôr a localidade português brasileiro
@*\sel{$\backslash$usepackage[utf8]\{inputenc\}}*@	% Comando para reconhecer entradas utilizando a codificação UTF-8

\begin{document}
(@*\ldots*@)
\end{lstlisting}
\ownsrc
\end{figure}

\begin{itemize}
	\item \sel{$\backslash$usepackage[utf8]\{inputenc\}}
	\begin{itemize}
		\item Podem ser usados outros tipos de codificação, como \texttt{latin1} em vez de \texttt{utf8}, a qual corresponde à codificação ISO 8859-1
		\item \textbf{Recomendação pessoal}: Usar UTF-8
	\end{itemize}
\end{itemize}
\end{frame}







