\section{Outras Possibilidades}

\begin{frame}[fragile] \frametitle{Outras Possibilidades}
\begin{itemize}
	\item Existem \textbf{muitas} possibilidades em Latex
	\begin{itemize}
		\item Comunidade do Latex bem ativa
		\item Desenvolvimento de diversos pacotes para múltiplos fins
	\end{itemize}
\end{itemize}
\end{frame}

\begin{frame}[fragile] \frametitle{Apresentações}
\begin{itemize}
	\item Mudanças na \textbf{classe de documentos}
	\item Alguns comandos adicionais
	\item De restante, \textbf{grande parte} dos comandos em Latex são aplicáveis
	\item Tal qual o \textit{template} do TSI para TCC, existe um formato voltado para apresentações
	\begin{itemize}
		\item Também disponível no github
		\item \url{https://github.com/gdotorres/apresentacao-tsi-pelotas}
	\end{itemize}
\end{itemize}
\end{frame}

\begin{frame}[fragile] \frametitle{Ambiente matemático}
\begin{itemize}
	\item Permite uma ampla variedade de comandos os quais fornecem meios para formatação de conceitos matemáticos
	\begin{itemize}
		\item Equações, fórmulas, \ldots
	\end{itemize}
	\item Utiliza uma série de caracteres especiais, os quais são definidos em uma extensa lista de comandos\footnote{\url{https://en.wikibooks.org/wiki/LaTeX/Mathematics\#List\_of\_mathematical\_symbols}}
	\item Possível construir equação em um ambiente online, permitindo mais rapidez e geração automática do código
	\begin{itemize}
		\item Online LaTeX Equation Editor: \url{https://www.codecogs.com/eqnedit.php}
	\end{itemize}
\end{itemize}
\end{frame}

\begin{frame}[fragile] \frametitle{Outras funcionalidades}
\begin{itemize}
	\item Desenho de figuras
	\item Gráficos
	\item Códigos
	\item Diversos
\end{itemize}
\end{frame}


