\section{Secionamento}

\begin{frame}[fragile] \frametitle{Secionamento}
\begin{itemize}
	\item A fim de promover uma organização melhor no texto, utilizamos seções
	\item Estas seções são \textbf{usualmente} numeradas
	\item Caso de exemplo: \textbf{Template do TCC-TSI}
	\begin{itemize}
		\item Cinco níveis de seção
		\item Seção primária (1), seção secundária (1.1), \ldots, seção quinária (1.1.1.1.1)
	\end{itemize}
	\item Numeração das seções: feita \textbf{automaticamente}
\end{itemize}
\end{frame}

\begin{frame}[fragile] \frametitle{Secionamento}
\begin{table}[!t]
\centering
\begin{tabular}{l|l}
\hline
	\textbf{Nível da seção} & \textbf{Comando}                      \\ \hline
	Primária                & \texttt{\textbackslash{}chapter\{nome\}}       \\ \hline
	Secundária              & \texttt{\textbackslash{}section\{nome\}}       \\ \hline
	Terciária               & \texttt{\textbackslash{}subsection\{nome\}}    \\ \hline
	Quaternária             & \texttt{\textbackslash{}subsubsection\{nome\}} \\ \hline
	Quinária                & \texttt{\textbackslash{}paragraph\{nome\}}     \\ \hline
\end{tabular}
\end{table}

\begin{itemize}
	\item As normas do IFSul nos restringem a no máximo seções quinárias
	\item Usualmente estes comandos são redefinidos nas classes utilizadas
\end{itemize}
\end{frame}

\begin{frame}[fragile] \frametitle{Secionamento não numerado}
\begin{itemize}
	\item Eventualmente, pode-se haver a necessidade de estruturar uma seção no documento não numerada
	\item Para isto, fazemos o uso do caractere \text{asterisco} (*) para tal fim
	\begin{itemize}
		\item Exemplo: \texttt{\textbackslash{}section*\{seção não numerada\}}
	\end{itemize}
\end{itemize}
\end{frame}

\begin{frame}[fragile] \frametitle{Secionamento (outras informações)}
\begin{itemize}
	\item Outras classes de documentos podem definir diferentes estilos de secionamento
	\item Dentro da linguagem LaTeX são fornecidos até 7 níveis de secionamento
\end{itemize}
\end{frame}

\begin{frame}[fragile] \frametitle{Secionamento (outras informações)}
\begin{table}[!t]
\centering
\begin{tabular}{l|l}
\hline
	\textbf{Nível da seção} & \textbf{Comando}									\\ \hline
	Primária				& \texttt{\textbackslash{}part\{nome\}}				\\ \hline
	Secundária				& \texttt{\textbackslash{}chapter\{nome\}}			\\ \hline
	Terciária				& \texttt{\textbackslash{}section\{nome\}}			\\ \hline
	Quaternária				& \texttt{\textbackslash{}subsection\{nome\}}		\\ \hline
	Quinária				& \texttt{\textbackslash{}subsubsection\{nome\}}	\\ \hline
	Senária					& \texttt{\textbackslash{}paragraph\{nome\}}		\\ \hline
	Setenária				& \texttt{\textbackslash{}subparagraph\{nome\}}		\\ \hline
\end{tabular}
\end{table}

\begin{itemize}
	\item Se realmente quiséssemos ir a fundo, podíamos definir outros comandos para estabelecer \textbf{mais níveis de secionamento}
\end{itemize}
\end{frame}

\begin{frame}[fragile] \frametitle{Referência à seções}
\begin{itemize}
	\item Eventualmente, podemos chegar a um ponto que precisamos fazer referência à uma determinada seção
	\item Exemplo clássico: Dentro da introdução de um TCC, falamos sobre como as demais seções estão organizadas no documento.
	\item Esta referenciação pode ser feita de maneira \textbf{automática}
	\item Utilizaremos um par de comandos para tal fim: \texttt{label} e \texttt{ref}
\end{itemize}
\end{frame}

\begin{frame}[fragile] \frametitle{Referência à seções}
\begin{figure}[!t]
\begin{lstlisting}
	\section{Laranjas} @*\sel{\textbackslash{}label\{s:laranjas\}}*@
	Laranja é laranja por causa da cor laranja ou por causa da fruta laranja?

	\section{Maracujás} @*\sel{\textbackslash{}label\{s:maracujas\}}*@
	Uma questão foi feita na Seção~\ref{s:laranjas}. Os maracujás tem a cor amarela.
\end{lstlisting}
\end{figure}

\begin{itemize}
	\item \sel{\textbackslash{}label\{nome do rótulo\}}
	\begin{itemize}
		\item Rotula um elemento em LaTeX. Neste caso, estamos utilizando para uma \texttt{section}
	\end{itemize}
\end{itemize}
\end{frame}

\begin{frame}[fragile] \frametitle{Referência à seções}
\begin{figure}[!t]
\begin{lstlisting}
\section{Laranjas} \label{s:laranjas}
Laranja é laranja por causa da cor laranja ou por causa da fruta laranja?

\section{Maracujás} \label{s:maracujas}
Uma questão foi feita na Seção~@*\sel{\textbackslash{}ref\{s:laranjas\}}*@. Os maracujás tem a cor amarela.
\end{lstlisting}
\end{figure}

\begin{itemize}
	\item \sel{\textbackslash{}ref\{nome do rótulo\}}
	\begin{itemize}
		\item Referencia um elemento em LaTeX previamente rotulado. O LaTeX encarrega-se de mostrar a numeração (neste caso) da seção corretamente
	\end{itemize}
\end{itemize}
\end{frame}