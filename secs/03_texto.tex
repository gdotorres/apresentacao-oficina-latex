\section{Texto}

\begin{frame}[fragile] \frametitle{Inserção de texto}
\begin{itemize}
	\item Para inserir texto, é extremamente simples
	\item Basta digitá-lo dentro do corpo de documento
	\item Entre \texttt{$\backslash$begin\{document\}} e \texttt{$\backslash$end\{document\}}
\end{itemize}
\end{frame}

\begin{frame}[fragile] \frametitle{Inserção de texto}
\begin{figure}[!t]
\caption{Exemplo de digitação simples de texto em Latex}
\begin{lstlisting}
(@*\ldots*@)
\begin{document}
(@*\ldots*@)
	Escrever um texto em LaTeX é simples como testar todos os caracteres como "The quick brown fox jumps over the lazy dog".
(@*\ldots*@)
\end{document}
\end{lstlisting}
\ownsrc
\end{figure}
\end{frame}

\begin{frame}[fragile] \frametitle{Inserção de texto}
\begin{itemize}
	\item O próprio LaTeX acha o melhor meio de fazer o texto caber dentro das margens
	\begin{itemize}
		\item Utiliza o espaçamento adequadamente
		\item Consegue fazer a quebra de palavras utilizando separação de sílabas para melhor disposição do texto em uma linha
	\end{itemize}
	\item Porém um texto pode apresentar outras características
	\begin{itemize}
		\item Como fazer um novo parágrafo?
		\item Uma nova linha?
	\end{itemize}
\end{itemize}
\end{frame}

\begin{frame} \frametitle{Parágrafos}
\begin{itemize}
	\item Um método para adição de parágrafos
	\begin{itemize}
		\item Inserção de uma linha em branco entre dois parágrafos.
	\end{itemize}
\end{itemize}
\end{frame}

\begin{frame}[fragile] \frametitle{Parágrafos}
\begin{figure}[!t]
\caption{Exemplo de inserção de parágrafos}
\begin{lstlisting}
(@*\ldots*@)
\begin{document}
(@*\ldots*@)
	Escrever um texto em LaTeX é simples como testar todos os caracteres como "The quick brown fox jumps over the lazy dog".

	Existe uma linha em branco acima da primeira linha, logo esta linha será um novo parágrafo.
	Note que esta frase não aparece em um novo parágrafo, já que não existe uma linha em branco entre a frase anterior e esta.
(@*\ldots*@)
\end{document}
\end{lstlisting}
\ownsrc
\end{figure}
\end{frame}

\begin{frame}[fragile] \frametitle{Nova linha}
\begin{itemize}
	\item Diferentes métodos para adição de uma nova linha
	\begin{enumerate}
		\item Método usual: comando $\backslash\backslash$
		\item Método alternativo: comando $\backslash$newline
	\end{enumerate}
\end{itemize}
\end{frame}

\begin{frame}[fragile] \frametitle{Inserção de texto}
\begin{figure}[!t]
\caption{Exemplo de nova linha pelo comando $\backslash\backslash$}
\begin{lstlisting}
(@*\ldots*@)
\begin{document}
(@*\ldots*@)
	Escrever um texto em LaTeX é simples como testar todos os caracteres como "The quick brown fox jumps over the lazy dog". \\ Existem duas contrabarras logo após a primeira frase, logo esta frase aparece em uma nova linha.
(@*\ldots*@)
\end{document}
\end{lstlisting}
\ownsrc
\end{figure}

\begin{itemize}
	\item Nova linha \textbf{não é} novo parágrafo
	\begin{itemize}
		\item Parágrafo tem o recuo à direita em relação a uma nova linha
	\end{itemize}
\end{itemize}
\end{frame}

	