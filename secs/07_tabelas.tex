\section{Tabelas}

\begin{frame}[fragile] \frametitle{Tabelas}
\begin{itemize}
	\item Comumente utilizamos uma tabela para fazer a sumarização de algumas informações
	\begin{itemize}
		\item Comparar informações
		\item Mostra de dados
	\end{itemize}
	\item Um elemento não tão simples de ser inserido em LaTeX
\end{itemize}
\end{frame}

\begin{frame}[fragile] \frametitle{Tabelas}
\begin{figure}[!t]
\begin{lstlisting}
\begin{table}[!h]
\centering
\caption{legenda}
	\begin{tabular}{|c|c|} \hline
		a & b \\ \hline
		c & d \\ \hline
	\end{tabular}
\source{fonte}
\end{table}
\end{lstlisting}
\end{figure}
\begin{table}[!h]
\centering
\begin{tabular}{|c|c|} \hline
	a & b \\ \hline
	c & d \\ \hline
\end{tabular}
\end{table}
\end{frame}

\begin{frame}[fragile] \frametitle{O ambiente \texttt{table}}
\vspace{-0.5cm}
\begin{figure}[!t]
\begin{lstlisting}
@*\sel{\textbackslash{}begin\{table\}[!h]}*@
\centering
\caption{legenda}
	\begin{tabular}{|c|c|} \hline
		a & b \\ \hline
		c & d \\ \hline
	\end{tabular}
\source{fonte}
@*\sel{\textbackslash{}end\{table\}}*@
\end{lstlisting}
\end{figure}

\begin{itemize}
	\item \sel{\textbackslash{}begin\{table\}} e \sel{\textbackslash{}end\{table\}}
	\begin{itemize}
		\item Define um ambiente de tabela
		\begin{itemize}
			\item Em especial, elementos como tabela e figura são chamados no LaTeX de \textit{floats}
		\end{itemize}
		\item Pode ter como argumento opcional o \textbf{posicionamento} do \textit{float} dentro do documento
	\end{itemize}
\end{itemize}
\end{frame}

\begin{frame}[fragile] \frametitle{Posicionamento de \textit{floats}}
\begin{itemize}
	\item Definido logo após um comando \texttt{\textbackslash{}begin\{"float"\}}
	\begin{itemize}
		\item Exemplo: \texttt{\textbackslash{}begin\{table\}[!h]}
	\end{itemize}
	\item Indica um ou mais possíveis posicionamentos do \textit{float} no LaTeX
\end{itemize}

\begin{table}[!t]
\label{tab:posicaofloats}
\begin{tabular}{c|p{7.5cm}} \hline
\textbf{Caractere} & \multicolumn{1}{c}{\textbf{Significado}} \\ \hline
h & Posiciona "aqui", sendo colocado aproximadamente no mesmo ponto que ocorre no texto.            \\ \hline
t & Posiciona ao topo da página             \\ \hline
b & Posiciona na base da página             \\ \hline
p & Insere em uma página especial somente para floats     \\ \hline
! & Sobrepõe os parâmetros internos que o LaTeX utiliza para determinar "boas"~posições para floats \\ \hline
\end{tabular}
\end{table}
\end{frame}

\begin{frame}[fragile,label={slide:floatcentering}] \frametitle{Centralização de \textit{floats}}
\vspace{-0.5cm}
\begin{figure}[!t]
\begin{lstlisting}
\begin{table}
@*\sel{\textbackslash{}centering}*@
\caption{legenda}
	\begin{tabular}{|c|c|} \hline
		a & b \\ \hline
		c & d \\ \hline
	\end{tabular}
\source{fonte}
\end{table}
\end{lstlisting}
\end{figure}

\begin{itemize}
	\item \sel{\textbackslash{}centering}
	\begin{itemize}
		\item Denota que o \textit{float} utilizado seja centralizado no documento
	\end{itemize}
\end{itemize}
\end{frame}

\begin{frame}[fragile,label={slide:floatcaptionsource}] \frametitle{Legenda e fonte de \textit{floats}}
\begin{figure}[!t]
\begin{lstlisting}
\begin{table}
\centering
@*\sel{\textbackslash{}caption\{legenda\}}*@
	\begin{tabular}{|c|c|} \hline
		a & b \\ \hline
		c & d \\ \hline
	\end{tabular}
@*\seli{\textbackslash{}source\{fonte\}}*@
\end{table}
\end{lstlisting}
\end{figure}

\begin{itemize}
	\item \sel{\textbackslash{}caption\{legenda\}}
	\begin{itemize}
		\item Legenda do \textit{float}
	\end{itemize}
	\item \seli{\textbackslash{}source\{fonte\}}
	\begin{itemize}
		\item Fonte do \textit{float}
		\item Este comando é definido \textbf{somente} na classe do TCC do TSI
	\end{itemize}
\end{itemize}
\end{frame}

\begin{frame}[fragile,label={slide:floatjustsource}] \frametitle{Fontes}
\begin{itemize}
	\item No caso da classe do TCC do TSI, o comando \texttt{\textbackslash{}source\{fonte\}} pode ser utilizado devido à uma linha de código na classe do documento:
	\begin{itemize}
		\item \texttt{\textbackslash{}newcommand\{\textbackslash{}source\}[1]\{\textbackslash{}\textbackslash{} Fonte: \#1\}}
	\end{itemize}
	\item Se adicionarmos tal linha de comando no preâmbulo de nosso documento, poderemos utilizar \texttt{\textbackslash{}source}
	\item Mas o que é esta linha de comando \ldots?
	\begin{itemize}
		\item De maneira sucinta, estamos criando um novo comando chamado \texttt{\textbackslash{}source}, o qual requer um argumento. Este comando faz uma quebra de linha, adiciona o texto "Fonte: "~e utiliza o argumento fornecido.
		\item Ao chamarmos \texttt{\textbackslash{}source\{Autoria própria\}}, seria a mesma coisa que inserir \texttt{{\textbackslash{}\textbackslash{} Fonte: Autoria própria}}
		\end{itemize}	
	\end{itemize}
\end{frame}

\begin{frame}[fragile] \frametitle{O ambiente \texttt{tabular}}
\begin{figure}[!t]
\begin{lstlisting}
\begin{table}
\centering
\caption{legenda}
	@*\sel{\textbackslash{}begin\{tabular\}}*@{|c|c|} \hline
		a & b \\ \hline
		c & d \\ \hline
	@*\sel{\textbackslash{}end\{tabular\}}*@
\source{fonte}
\end{table}
\end{lstlisting}
\end{figure}

\begin{itemize}
	\item \sel{\textbackslash{}begin\{tabular\}} e \sel{\textbackslash{}end\{tabular\}}
	\begin{itemize}
		\item Define um ambiente do tipo \textit{tabular}
		\begin{itemize}
			\item Permite o desenho/formatação de tabelas
		\end{itemize}
	\end{itemize}
\end{itemize}
\end{frame}

\begin{frame}[fragile] \frametitle{Parâmetros de \texttt{tabular}}
\begin{figure}[!t]
\begin{lstlisting}
\begin{table}
\centering
\caption{legenda}
	\begin{tabular}@*\sel{\{|c|c|\}}*@ \hline
		a & b \\ \hline
		c & d \\ \hline
	\end{tabular}
\source{fonte}
\end{table}
\end{lstlisting}
\end{figure}

\begin{itemize}
	\item O ambiente \texttt{tabular} requer um argumento obrigatório, o qual indica vários parâmetros da tabela
	\begin{enumerate}
		\item Número de colunas da tabela
		\item Alinhamento do texto nas colunas da tabela
		\item Se a tabela apresenta linhas verticais
	\end{enumerate}
\end{itemize}
\end{frame}

\begin{frame}[fragile] \frametitle{Parâmetros de \texttt{tabular}}
\begin{figure}[!t]
\begin{lstlisting}
\begin{table}
\centering
\caption{legenda}
	\begin{tabular}{|@*\sel{c}*@|@*\sel{c}*@|}\hline
		a & b \\ \hline
		c & d \\ \hline
	\end{tabular}
\source{fonte}
\end{table}
\end{lstlisting}
\end{figure}

\begin{enumerate}
	\item Número de colunas da tabela
	\begin{itemize}
		\item Caso de exemplo: Temos \textbf{dois} caracteres alfabéticos. Assim, nossa tabela irá conter \textbf{duas} colunas
		\item Outro exemplo: \texttt{\{|l|c|c|r|\}}. Neste caso, há quatro caracteres. Logo, quatro colunas.
	\end{itemize}
\end{enumerate}

\end{frame}

\begin{frame}[fragile] \frametitle{Parâmetros de \texttt{tabular}}
\vspace{-0.5cm}
\begin{figure}[!t]
\begin{lstlisting}
\begin{table}
\centering
\caption{legenda}
	\begin{tabular}{|@*\sel{c}*@|@*\sel{c}*@|} \hline
		a & b \\ \hline
		c & d \\ \hline
	\end{tabular}
\source{fonte}
\end{table}
\end{lstlisting}
\end{figure}

\begin{enumerate}
	\setcounter{enumi}{1}
	\item Alinhamento do texto nas colunas da tabela
	\begin{itemize}
		\item Dependendo do caractere utilizado, este irá definir qual o alinhamento do texto dentro da tabela de cada coluna 
		\begin{itemize}
			\item 1 caractere = 1 coluna
		\end{itemize}
	\end{itemize}
\end{enumerate}
\end{frame}

\begin{frame}[fragile] \frametitle{Alinhamento de texto nas colunas das tabelas}

\begin{table}[!t]
\begin{tabular}{l|p{9cm}} \hline
\texttt{l}			& Coluna com texto justificado à esquerda \\ \hline
\texttt{c}			& Coluna com texto centralizado \\ \hline
\texttt{r}			& Coluna com texto justificado à direita \\ \hline
\texttt{p\{tam\}}	& Coluna com tamanho fixo definido em \texttt{tam}. Se o texto ultrapassar o tamanho definido, há uma quebra de linha no texto. O texto é verticalmente alinhado ao topo \\ \hline
\end{tabular}
\end{table}
\end{frame}

\begin{frame}[fragile] \frametitle{Parâmetros de \texttt{tabular}}
\begin{figure}[!t]
\begin{lstlisting}
\begin{table}
\centering
\caption{legenda}
	\begin{tabular}{@*\sel{|}*@c@*\sel{|}*@c@*\sel{|}*@} \hline
		a & b \\ \hline
		c & d \\ \hline
	\end{tabular}
\source{fonte}
\end{table}
\end{lstlisting}
\end{figure}

\begin{enumerate}
	\setcounter{enumi}{2}
	\item Presença ou ausência de linhas verticais entre as colunas
	\begin{itemize}
		\item Dois tipos de linhas verticais:
		\begin{itemize}
			\item \texttt{|}: linha vertical simples
			\item \texttt{||}: linha vertical dupla
		\end{itemize}
		\item Se não há especificação de linhas, elas não serão desenhadas na tabela
	\end{itemize}
\end{enumerate}
\end{frame}

\begin{frame}[fragile] \frametitle{Conteúdo da tabela}
\begin{figure}[!t]
\begin{lstlisting}
\begin{table}
\centering
\caption{legenda}
	\begin{tabular}{|c|c|} \hline
		@*\sel{a \& b}*@ \\ \hline
		c & d \\ \hline
	\end{tabular}
\source{fonte}
\end{table}
\end{lstlisting}
\end{figure}

\begin{itemize}
	\item Dentro do ambiente tabular definimos o conteúdo das tabelas
	\item Segue um formato:
	\begin{itemize}
		\item Texto da 1ª coluna \& Texto da 2ª coluna \& \ldots \& Texto da última coluna
	\end{itemize}
\end{itemize}
\end{frame}

\begin{frame}[fragile] \frametitle{Conteúdo da tabela}
\begin{figure}[!t]
\begin{lstlisting}
\begin{table}
\centering
\caption{legenda}
	\begin{tabular}{|c|c|} \hline
		a & b @*\sel{\textbackslash{}\textbackslash{}}*@ \hline
		c & d \\ \hline
	\end{tabular}
\source{fonte}
\end{table}
\end{lstlisting}
\end{figure}

\begin{itemize}
	\item Quando chegamos no texto da última coluna dentro de uma linha e desejamos inicializar uma nova linha, utilizamos o comando de nova linha (\texttt{\textbackslash{}\textbackslash{}})
\end{itemize}
\end{frame}

\begin{frame}[fragile] \frametitle{Conteúdo da tabela}
\begin{figure}[!t]
\begin{lstlisting}
\begin{table}
\centering
\caption{legenda}
	\begin{tabular}{|c|c|}@*\sel{\textbackslash{}hline}*@
		a & b \\ @*\sel{\textbackslash{}hline}*@
		c & d \\ @*\sel{\textbackslash{}hline}*@
	\end{tabular}
\source{fonte}
\end{table}
\end{lstlisting}
\end{figure}

\begin{itemize}
	\item Da mesma maneira que podemos inserir linhas verticais ou não, podemos fazê-la com linhas horizontais
	\item Utilizamos o comando \texttt{\textbackslash{}hline}
\end{itemize}
\end{frame}

\begin{frame}[fragile] \frametitle{Como fazer uma tabela destas?}
\begin{table}[!t]
\centering
\scalebox{0.85}{
\begin{tabular}{|c|l|l|l|}
\hline
\multicolumn{4}{|c|}{\textbf{\begin{tabular}[c]{@{}c@{}}Escalação da Seleção Brasileira\\ Brasil 1x2 Bélgica (06/07/2018)\end{tabular}}} \\ \hline
Goleiro & GOL & 1  & Alisson           \\ \hline
\multirow{4}{*}{Defesa}             & LAD & 22 & Fagner            \\ \cline{2-4} 
        & ZAG & 2  & Thiago Silva      \\ \cline{2-4} 
        & ZAG & 3  & Miranda           \\ \cline{2-4} 
        & LAE & 12 & Marcelo           \\ \hline
\multicolumn{1}{|l|}{\multirow{3}{*}{Meio-campo}} & VOL & 17 & Fernandinho       \\ \cline{2-4} 
\multicolumn{1}{|l|}{}& VOL & 15 & Paulinho          \\ \cline{2-4} 
\multicolumn{1}{|l|}{}& MEI & 11 & Philippe Coutinho \\ \hline
\multirow{3}{*}{Ataque}             & ATA & 19 & Willian           \\ \cline{2-4} 
        & ATA & 10 & Neymar            \\ \cline{2-4} 
        & CTA & 9  & Gabriel Jesus     \\ \hline
\end{tabular}
}
\end{table}
\end{frame}

\begin{frame}[fragile] \frametitle{Como fazer uma tabela destas?}

\begin{figure}[!t]
\begin{lstlisting}
\begin{table}[!t]
\centering
\caption{A escalação brasileira naquele dia}
\begin{tabular}{|c|l|l|l|} \hline
\multicolumn{4}{|c|}{\textbf{\begin{tabular}[c]{@{}c@{}}Escalação da Seleção Brasileira\\ Brasil 1x2 Bélgica (06/07/2018)\end{tabular}}} \\ \hline
Goleiro & GOL & 1 & Alisson \\ \hline
\multirow{4}{*}{Defesa} & LAD & 22 & Fagner \\ \cline{2-4} 
 & ZAG & 2 & Thiago Silva \\ \cline{2-4} 
 & ZAG & 3 & Miranda \\ \cline{2-4} 
 & LAE & 12 & Marcelo \\ \hline
(@*\ldots*@)
\end{lstlisting}
\end{figure}
\end{frame}

\begin{frame}[fragile] \frametitle{Como fazer uma tabela destas?}
\begin{figure}[!t]
\begin{lstlisting}
(@*\ldots*@)
\multicolumn{1}{|l|}{\multirow{3}{*}{Meio-campo}} & VOL & 17 & Fernandinho \\ \cline{2-4} 
\multicolumn{1}{|l|}{}& VOL & 15 & Paulinho \\ \cline{2-4} 
\multicolumn{1}{|l|}{}& MEI & 11 & Philippe Coutinho \\ \hline
\multirow{3}{*}{Ataque} & ATA & 19 & Willian \\ \cline{2-4} 
 & ATA & 10 & Neymar \\ \cline{2-4} 
 & CTA & 9 & Gabriel Jesus \\ \hline
\end{tabular}
\\ \vspace{0.2cm} \small{Fonte: Autoria própria}
\end{table}
\end{lstlisting}
\end{figure}
\end{frame}

\begin{frame}[fragile] \frametitle{Como fazer uma tabela destas?}
\begin{itemize}
	\item Complexidade de código aumenta substancialmente conforme mais recursos de tabela são utilizados
	\begin{itemize}
		\item Colunas que ocupam várias colunas, linhas que ocupam várias linhas, diferentes alinhamentos na mesma coluna em linhas diferentes, \ldots
	\end{itemize}
	\item Existem facilidades: Você constrói a tabela de maneira intuitiva e o código é gerado automaticamente
	\begin{itemize}
		\item \url{https://www.tablesgenerator.com/}
		\item \url{https://truben.no/table/}
	\end{itemize}
\end{itemize}
\end{frame}

\begin{frame}[fragile] \frametitle{Refência a tabelas}

\begin{itemize}
	\item Vimos como fazer refência à seções, e como o LaTeX consegue lidar com a numeração automaticamente
	\item Eventualmente um trabalho conterá tabelas as quais precisam ser explicadas durante o texto
	\item Tabelas são usualmente numeradas, e assim podemos referenciá-las
\end{itemize}

\end{frame}

\begin{frame}[fragile,label={slide:referenciatabela}] \frametitle{Refência a tabelas}
\begin{figure}[!t]
\begin{lstlisting}
\begin{table}
\centering
\caption{legenda}
@*\sel{\textbackslash{}label\{minhatabela\}}*@
	\begin{tabular} (@*\ldots*@) \end{tabular}
\source{fonte}
\end{table}

Na Tabela~@*\seli{\textbackslash{}ref\{minhatabela\}}*@, são mostrados (@*\ldots*@)
\end{lstlisting}
\end{figure}

\begin{itemize}
	\item Fazemos de maneira parecida com o que ocorre na referência à seções
	\item O comando \texttt{\textbackslash{}label} deve ir logo abaixo de \texttt{\textbackslash{}caption}
	\item Usamos \texttt{\textbackslash{}ref} para referenciar a tabela
\end{itemize}

\end{frame}