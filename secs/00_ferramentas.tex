\section{Ferramentas}

\begin{frame}[fragile] \frametitle{Como começamos?}
\begin{itemize}
	\item \textit{Software} essencial para construção de documentos em LaTeX
	\begin{itemize}
		\item Miktex\footnote{\url{https://miktex.org/}}
		\begin{itemize}
			\item Comumente utilizado no Windows
		\end{itemize}
		\item Texlive\footnote{\url{https://www.tug.org/texlive/}}
		\begin{itemize}
			\item Comumente utilizado em sistemas baseados em Linux
		\end{itemize}
	\end{itemize}
\end{itemize}
\end{frame}

\begin{frame}[fragile] \frametitle{Como começamos?}
\begin{itemize}
	\item Tendo um dos programas citados, podemos utilizar \textbf{qualquer} editor de texto
	\begin{itemize}
		\item Basta saber como compilar o arquivo fonte
	\end{itemize}
	\item Porém, existem alternativas
\end{itemize}
\end{frame}

\begin{frame}[fragile] \frametitle{IDEs}
\begin{itemize}
	\item Podemos utilizar IDEs que facilitam um pouco o trabalho
	\item IDEs contam com \textit{syntax highlighting}, métodos de compilação em um clique, facilidades para utilizar os comandos a serem vistos, dentre outros
	\begin{itemize}
		\item \textbf{Texmaker}~\footnote{\url{http://www.xm1math.net/texmaker/}} (recomendado)
		\begin{itemize}
			\item Funciona em Windows, Linux e MacOsX
			\item \textit{Cross-platform}
		\end{itemize}
		\item Texworks~\footnote{\url{http://www.tug.org/texworks/}}
		\item Kile~\footnote{\url{https://kile.sourceforge.io/}}
	\end{itemize}
\end{itemize}
\end{frame}

\begin{frame}[fragile] \frametitle{Via web}
\begin{itemize}
	\item Podemos usar editores web
	\item Apresentam suporte à escrita colaborativa
	\begin{itemize}
		\item Overleaf~\footnote{\url{https://www.overleaf.com/}}
		\item ShareLaTeX~\footnote{\url{https://www.sharelatex.com/}}
	\end{itemize}
\end{itemize}
\end{frame}