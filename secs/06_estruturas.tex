\section{Estruturas de Listas}

\begin{frame}[fragile] \frametitle{Estruturas de Listas}
\begin{itemize}
	\item Em alguns casos podemos fazer uso de estruturas de listas para enumerar uma sequência de itens
	\item Algo como está sendo visualizado nesta apresentação
	\item Existem três tipos de estruturas de listas
	\begin{itemize}
		\item Não numeradas
		\item Enumeradas
		\item Descritivas
	\end{itemize}
\end{itemize}
\end{frame}

\begin{frame}[fragile] \frametitle{Listas não numeradas}
\begin{figure}[!t]
\begin{lstlisting}
@*\sel{\textbackslash{}begin\{itemize\}}*@
	\item Um item
	\item Outro item
@*\sel{\textbackslash{}end\{itemize\}}*@
\end{lstlisting}
\end{figure}

\begin{itemize}
	\item \sel{\textbackslash{}begin\{itemize\}} e \sel{\textbackslash{}end\{itemize\}}
	\begin{itemize}
		\item Denomina um \textbf{ambiente} de nome \texttt{itemize}, o qual permite construir listas não numeradas
		\item Todo tipo de comando onde for utilizado neste estilo (\texttt{begin} e \texttt{end}) é chamado de ambiente em LaTeX
	\end{itemize}
\end{itemize}
\end{frame}

\begin{frame}[fragile] \frametitle{Listas não numeradas}
\begin{figure}[!t]
\begin{lstlisting}
\begin{itemize}
	@*\sel{\textbackslash{}item}*@ Um item
	@*\sel{\textbackslash{}item}*@ Outro item
\end{itemize}
\end{lstlisting}
\end{figure}

\begin{itemize}
	\item \sel{\textbackslash{}item}
	\begin{itemize}
		\item Determina um item de uma lista não numerada
		\item Este comando \textbf{deve} estar dentro de um ambiente que permite construir listas! (Ex: \texttt{itemize})
	\end{itemize}
\end{itemize}
\end{frame}

\begin{frame}[fragile] \frametitle{Listas enumeradas}
\begin{figure}[!t]
\begin{lstlisting}
@*\sel{\textbackslash{}begin\{enumerate\}}*@
	\item Primeiro item enumerado
	\item Segundo item enumerado
@*\sel{\textbackslash{}end\{enumerate\}}*@
\end{lstlisting}
\end{figure}

\begin{itemize}
	\item \sel{\textbackslash{}begin\{enumerate\}} e \sel{\textbackslash{}end\{enumerate\}}
	\begin{itemize}
		\item Denomina um ambiente de nome \texttt{enumerate}, o qual permite construir listas enumeradas
	\end{itemize}
\end{itemize}
\end{frame}

\begin{frame}[fragile] \frametitle{Listas enumeradas}
\begin{figure}[!t]
\begin{lstlisting}
\begin{enumerate}
	@*\sel{\textbackslash{}item}*@ Um item
	@*\sel{\textbackslash{}item}*@ Outro item
\end{enumerate}
\end{lstlisting}
\end{figure}

\begin{itemize}
	\item \sel{\textbackslash{}item}
	\begin{itemize}
		\item Determina um item de uma lista enumerada
		\begin{itemize}
			\item Neste caso é enumerada porque está \textbf{dentro do ambiente} \texttt{enumerate}
		\end{itemize}
		\item O LaTeX efetua a numeração da lista automaticamente
	\end{itemize}
\end{itemize}
\end{frame}

\begin{frame}[fragile] \frametitle{Listas enumeradas manipuladas}

\begin{itemize}
	\item A priori, temos listas com números
	\item Podemos ir além
	\item Uso de um pacote \texttt{enumitem}
	\begin{itemize}
		\item \texttt{\textbackslash{}usepackage\{enumitem\}}
		\item Permite fazer manipulação nas listas de modo a permitir listas ordenadas por caractere (a, b, \ldots), números romanos (i, ii, \ldots), dentre outros
	\end{itemize}
\end{itemize}
\end{frame}

\begin{frame}[fragile] \frametitle{Listas enumeradas manipuladas}
\begin{figure}[!t]
\begin{lstlisting}
(@*\ldots*@)
\usepackage{enumitem}
(@*\ldots*@)
\begin{document}
(@*\ldots*@)
\begin{enumerate}@*\sel{[label=\textbackslash{}roman*]}*@
	\item Um item
	\item Outro item
\end{enumerate}
\end{lstlisting}
\end{figure}

\begin{itemize}
	\item \sel{[label=formato]}
	\begin{itemize}
		\item Determina que uma lista enumerada utilize um determinado formato
		\item Neste caso, estamos utilizando \textbackslash{}roman*, o qual formata os itens de uma lista do tipo \texttt{enumerate} para números romanos
	\end{itemize}
\end{itemize}
\end{frame}

\begin{frame}[fragile] \frametitle{Listas enumeradas manipuladas}
\begin{itemize}
	\item Argumento para formatação de lista enumerada é opcional
	\begin{itemize}
		\item Qualquer argumento que vier dentro de colchetes é considerado pela linguagem do LaTeX como argumento opcional
	\end{itemize}
	\item Se nada é passado para \texttt{\textbackslash{}begin\{enumerate\}} o padrão é utilizado (lista enumerada numericamente)
\end{itemize}
\end{frame}

\begin{frame}[fragile] \frametitle{Formatos de listas enumeradas}
\begin{table}[!h]
\centering
\begin{tabular}{l|l}		
\hline
	\textbf{Comando}       & \textbf{Formato} \\ \hline
	\textbackslash{}alph   & a. b. c.         \\ \hline
	\textbackslash{}Alph   & A. B. C          \\ \hline
	\textbackslash{}arabic & 1. 2. 3.         \\ \hline
	\textbackslash{}roman  & i. ii. iii.      \\ \hline
	\textbackslash{}Roman  & I. II. III       \\ \hline
\end{tabular}
\end{table}
\end{frame}

\begin{frame}[fragile] \frametitle{Listas descritivas}
\begin{figure}[!t]
\begin{lstlisting}
@*\sel{\textbackslash{}begin\{description\}}*@
	\item [Banana] Uma fruta
	\item [Maçã] Outra fruta
@*\sel{\textbackslash{}end\{description\}}*@
\end{lstlisting}
\end{figure}

\begin{itemize}
	\item \sel{\textbackslash{}begin\{description\}} e \sel{\textbackslash{}end\{description\}}
	\begin{itemize}
		\item Denomina um ambiente de nome \texttt{description}, o qual permite construir listas descritivas
	\end{itemize}
\end{itemize}
\end{frame}

\begin{frame}[fragile] \frametitle{Listas descritivas}
\begin{figure}[!t]
\begin{lstlisting}
\begin{description}
	@*\sel{\textbackslash{}item [Banana]}*@ Uma fruta
	@*\sel{\textbackslash{}item [Maçã]}*@ Outra fruta
\begin{description}
\end{lstlisting}
\end{figure}

\begin{itemize}
	\item \sel{\textbackslash{}item}
	\begin{itemize}
		\item Determina um item de uma lista descritiva
		\begin{itemize}
			\item Dentro dos colchetes vai a descrição de um item
			\item Comumente formatado em negrito
		\end{itemize}
	\end{itemize}
\end{itemize}
\end{frame}

\begin{frame}[fragile] \frametitle{\textit{Mix} de listas}
\begin{figure}[!t]
\begin{lstlisting}
\begin{itemize}
	\item Um item não numerado
	\begin{enumerate}
		\item Item enumerado-x
		\begin{enumerate}
			\item Item enumerado-y
		\end{enumerate}
	\end{enumerate}
\end{itemize}
\end{lstlisting}
\end{figure}

\begin{itemize}
	\item Podemos inserir listas dentro de listas
	\item Somente deve-se tomar cuidado em que nível estamos ao adicionar um item
\end{itemize}
\end{frame}